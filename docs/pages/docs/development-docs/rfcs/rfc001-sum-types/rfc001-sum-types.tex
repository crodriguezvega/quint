% Created 2023-08-05 Sat 00:09
% Intended LaTeX compiler: xelatex
\documentclass[11pt]{article}
\usepackage{graphicx}
\usepackage{longtable}
\usepackage{wrapfig}
\usepackage{rotating}
\usepackage[normalem]{ulem}
\usepackage{amsmath}
\usepackage{amssymb}
\usepackage{capt-of}
\usepackage{hyperref}
\author{Shon Feder}
\date{\textit{<2023-08-03 Thu>}}
\title{RFC 001: Extend Quint with Row-Polymorphic Sum Types}
\hypersetup{
 pdfauthor={Shon Feder},
 pdftitle={RFC 001: Extend Quint with Row-Polymorphic Sum Types},
 pdfkeywords={},
 pdfsubject={},
 pdfcreator={Emacs 28.1 (Org mode 9.6.1)}, 
 pdflang={English}}
\begin{document}

\maketitle
\tableofcontents


\section{Overview}
\label{sec:orgabf3583}
This section gives a concise overview of the principle proposal. In
depth discussion follows in \hyperref[sec:org050246d]{Discuisson}.

\subsection{Concrete syntax}
\label{sec:orga6e64d7}
The concrete syntax falls into three categories, corresponding three aspects of
the semantics:

\begin{itemize}
\item Declaring that a type is allowed in a context.
\item Constructing an element of the declared type.
\item Eliminating an element of the declared type (to derive an element of some
other type).
\end{itemize}

Here we merely state the proposed syntax. See \hyperref[sec:orgb377490]{the discussion of concrete syntax}
for consideration of motivations and trade offs.

\subsubsection{Declaration}
\label{sec:orgf216e02}
The proposed syntax fits complements our syntax for records and follows the
precedent of most languages that include support for general sum-types:

\begin{verbatim}
type T =
  | A(int)
  | B(str)
  | C
\end{verbatim}

As indicated by \texttt{C}, labels with no associated expression type are allowed.
These are sugar for a label associated with the unit type (i.e., the empty
record). The above is therefore a representation of

\begin{verbatim}
type T =
  | A(int)
  | B(str)
  | C({})
\end{verbatim}

\subsubsection{Constructors}
\label{sec:orgc7e9737}

Internally, we need to add an operator \texttt{variant} similar to the ``injection''
operator from \href{https://www.microsoft.com/en-us/research/publication/extensible-records-with-scoped-labels/}{Leijen05}:

\begin{verbatim}
<l = _> :: ∀αr . α → <l :: α | r > -- injection
\end{verbatim}


This operation injects an expression of type \texttt{α} into a sum-type that includes a
variant associating label \texttt{l} with type \texttt{α}.  However, if we don't want to
expose the row-polymorphism to users, we'll need a more restrictive typing. This
is discussed in the section on typing rules.

We provide syntax sugar for users: when a sum type declaration is parsed,
constructor operations for each variant will be generated internally. E.g., for
the type \texttt{T} defined above, we will add to the context the following operators:

\begin{verbatim}
pure def A(x:int): T = variant(A, x)
pure def B(x:str): T = variant(B, x)
pure def C(): T      = variant(C, {})
\end{verbatim}

\subsubsection{Case analyses}
\label{sec:orgc4b162c}
Expressions of a sum type can be eliminated using a case analysis via
a \texttt{match} statement. Following Rust's syntax, we write that as

\begin{verbatim}
match e {
  | A(a) => ...
  | B(b) => ...
  | C    => ...
}
\end{verbatim}

This construct can either be a primitive or syntax sugar for a more
primitive \texttt{decompose} operator, discussed below.

We could also consider a case analysis \texttt{\{ A(a) => e1, B(b) => e2, C => e3 \}} --
where \texttt{e1, e2, e3 : S} -- as syntax sugar for an anonymous operator of type \texttt{T
=> S}. \texttt{match} would then be syntax sugar for operator application. This follows
Scala's case blocks or OCaml's \texttt{function} keyword and would allow idioms such
as:

\begin{verbatim}
setOfTs.map({ A(a) => e1 | B(b) => e2 | C => e3 })
\end{verbatim}

instead of requiring

\begin{verbatim}
setOfTs.map(x => match x { A(a) => e1| B(b) => e2| C => e3 })
\end{verbatim}

\subsection{Statics}
\label{sec:org04cad71}
These type rules assume we keep our current approach of using quint
strings for labels. But see my argument for simplifying our approach
under \hyperref[sec:orgd5aba72]{Drop the exotic operators}. See the
discussion in \hyperref[sec:org6da5dcb]{Statics} below for a detailed explanation and
analysis.

\subsubsection{Construction}
\label{sec:org23575b1}
The typing rule for constructing a variant of a sum type:

$$
\frac
{
\Gamma \vdash e \colon (t, c) \quad
\Gamma \vdash `l' \colon str \quad
definedType(s) \quad
free(v)
}
{
\Gamma \vdash \ variant(`l', e) \ :
(s, c \land s \sim \langle \ l \colon t | v \ \rangle)
}
$$


This rule is substantially different from \href{https://www.microsoft.com/en-us/research/publication/extensible-records-with-scoped-labels/}{Leijen05}'s

\begin{verbatim}
<l = i> :: ∀αr . α → <l :: α | r > -- injection
\end{verbatim}


because we have decided not to expose the underlying row-polymorphism for sum
types at this point. This introduces the non-trivial complication of needing to
introduce the typing context onto our judgments (this is what we gesture at with
the side condition \texttt{definedType(s)}).

\subsubsection{Elimination}
\label{sec:orgf258cbd}

The typing rule for eliminating a variant of a sum type via case
analysis:

$$
\frac{
\Gamma \vdash e : (s, c) \quad
\Gamma, x_1 \vdash e_1 : (t, c_1) \quad \ldots \quad \Gamma, x_n \vdash e_n : (t, c_n) \quad
\Gamma, \langle v \rangle \vdash e_{n+1} : (t, c_{n+1}) \quad
fresh(v)
}{
\Gamma \vdash \ match \ e \ \{ i_1 : x_1 \Rightarrow e_1, \ldots, i_n : x_n \Rightarrow e_n \} : (t,
c \land c_1 \land \ldots \land c_n \land c_{n+1} \land
s \sim \langle i_1 : t_1, \ldots, i_n : t_n | v \rangle)
}
$$


This gives a rule in our system that is sufficient to capture \href{https://www.microsoft.com/en-us/research/publication/extensible-records-with-scoped-labels/}{Leijen05}'s

\begin{verbatim}
(l ∈ _ ? _ : _) :: ∀αβr . <l :: α | r> → (α → β) → (<r> → β) → β -- decomposition
\end{verbatim}

since we can define decomposition for any label \texttt{L} via

\begin{verbatim}
def decomposeL(e: (L(a) | r), f: a => b, default : r => b) = 
  match e { 
    | L(x) => f(x) 
    | r    => default(r) 
  }
\end{verbatim}

However we can define \texttt{match} as syntax sugar for the decompose
primitive if we prefer.

\subsection{Dynamics}
\label{sec:orgd77d8b6}
The dynamics in the simulator should be straightforward and is not
discussed here. Translation to Apalache for symbolic execution in the
model checker is also expected to be relatively straight forward, since
Apalache has a very similar form of row-based sum typing.

The general rules for eager evaluation can be found in
\href{https://www.cs.cmu.edu/\~rwh/pfpl.html}{PFPL}, section 11.2.
Additional design work for this will be prepared if needed.

\noindent\rule{\textwidth}{0.5pt}

This concludes the tl;dr overview of the proposal. The remaining is an
indepth (still v. rough in places, discussion).

\section{Discussion}
\label{sec:org050246d}
\subsection{Motivation}
\label{sec:org2103e46}
Quint's type system currently supports product types. Product types
(i.e., records, with tuples as a special case where fields are indexed
by an ordinal) let us specify \emph{conjunctions} of data types in a way that
is verifiable statically. This lets us describe more complex data
structures in terms of values of specific types that \textbf{must} be packaged
together. E.g., we might define a rectangle by its length and width and
a triangle by the lengths of its three sides. Using Quint's existing
syntax for product types, we'd specify this as follows:

\begin{verbatim}
type Rectangle = 
  { l : int
  , w : int }
type Triangle = 
  { a : int
  , b : int
  , c : int }
\end{verbatim}

Quint's type system does not yet have the the dual construct,
\href{https://en.wikipedia.org/wiki/Tagged\_union}{sum types} (aka
``variants'', ``co-products'', or ``tagged unions''). Sum types specify
\emph{disjunctions} of data types in a way that is verifiable statically.
This lets us describe mutually exclusive alternatives between distinct
data structures that \textbf{may} occur together and be treated uniformly in
some context. E.g., we might wish to specify a datatype for shapes, so
we can work with collections that include both rectangles and triangles.
Using one of the proposed syntax option that will be motivated in the
following, this could be specified as

\begin{verbatim}
type Shape =
  | Rect(rectangle)
  | Tri(triangle)
\end{verbatim}

Having both product types and sum types (co-product types) gives us a
simple and powerful idiom for specifying families of data structures:

\begin{itemize}
\item We describe \emph{what must be given together} to form a product of the
specified type, and so \emph{what we may always make use of} by projection
when we are given such a product.
\item We describe \emph{which alternatives may be supplied} to form a co-product
of a specified type, and so \emph{what we must be prepared to handle}
during case analysis when we are given such a co-product.
\end{itemize}

E.g., a \texttt{rectangle} is defined by \emph{both} a length \emph{and} a width,
packaged together, while a \texttt{shape} is defined \emph{either} by a rectangle
\emph{or} a triangle. With these definitions established, we can then go on
to form and reason about collections of shapes like \texttt{Set[shape]}, or
define properties common to all shapes like
\texttt{isEquilateral : shape => bool}\footnote{The expressive power of these simple constructs comes from the
nice algebraic properties revealed when values of a type are
treated as equal up-to ismorphism. See, e.g.,
\url{https://codewords.recurse.com/issues/three/algebra-and-calculus-of-algebraic-data-types}}.

\subsection{Context}
\label{sec:orgea625f3}
\subsubsection{Existing plans and previous work}
\label{sec:orgc77cf04}
We have always planned to support co-products in quint: their utility is well
known and widely appreciated by engineers with experience in modern programming
languages. We introduced co-products to Apalache in
\url{https://github.com/apalache-mc/apalache/milestone/60} for the same reasons.
The design and implementation of the latter was worked out by ?? (????)
based on the paper \href{https://www.microsoft.com/en-us/research/publication/extensible-records-with-scoped-labels/}{``Extensible Records with Scoped Labels''}. At the core of this
design is a simple use of row-polymorphism that enables both extensible variants
and extensible records, giving us products and co-products in a one neat
package. The quint type system was also developed using row-polymorphism
following this design. As a result of this forethought, extension of quint's
type system and addition of syntax to support sum-types is expected to be
relatively straightforward.

\subsubsection{The gist of extensible row-typed records and sum types}
\label{sec:org6428ac3}
The core concept in the row-based approach we've opted for is the
following: we can use the same construct, called a ``row'', to represent
the \emph{conjoined} labeled fields of a product type and the \emph{alternative}
labeled choices of a sum type. That the row types are polymorphic lets
us extend the products and sums using row variables.

E.g., given the row

$$
i_1 : t_1 \ , \ldots \ , i_n : i_n | v
$$

with each \(t_k\)-typed field indexed by label \(i_k\) for
\(1 \le k \le n\) and the free row variable \(v\), then

$$
\{i_1 : t_1 \ , \ldots \ , i_n : i_n | v\}
$$

is an open record conjoining the fields, and

$$
\langle i_1 : t_1 \ , \ldots \ , i_n : i_n | v \rangle
$$

is an open sum type presenting the fields as (mutually exclusive)
alternatives. Both types are extensible by substituting \(v\) with
another (possibly open row). To represent a closed row, we omit the
trailing \(| v\).

\subsubsection{Quint's current type system}
\label{sec:org12dc16c}

The \href{https://github.com/informalsystems/quint/tree/main/doc/adr005-type-system.md}{current type system supported by quint} is based on a simplified version of
the constraint-based system presented in \href{https://www.microsoft.com/en-us/research/publication/complete-and-decidable-type-inference-for-gadts/}{``Complete and Decidable Type Inference
for GADTs''} augmented with extensible (currently, just) records based on
``Extensible Records with Scoped Labels''. A wrinkle in this genealogy is that
quint's type system includes neither GADTs nor scoped labels (and even the
extensiblity supported for records is limited). Moreover, due to their
respective foci, neither of the referenced papers includes a formalization the
complete statics for product types or sum types, and while we have implemented
support for product types in quint, we don't have our typing rules recorded.

\subsection{Statics}
\label{sec:org6da5dcb}
This section discusses the typing judgements that will allow us to
statically verify correct introduction and elimination of expressions
that are variants of a sum type. The following considerations have
informed the structure in which the proposed statics are discussed:

\begin{itemize}
\item Since sum-types are dual to product types, I consider their
complementary typing rules together: first I will present the relevant
rule for product types, then propose the complementary rule for sum
types. This should help maintain consistency between the two kinds of
typing judgements and ensure our implementations of both harmonize.
\item Since we don't have our existing product formation or elimination
rules described separate from the implementation, transcribing them
here can serve to juice our intuition, supplement our design
documentation, and perhaps give opportunity for refinement.
\item Since our homegrown type system has some idiosyncrasies that can
obscure the essence of the constructs under discussion, I precede the
exposition of each rule with a text-book example adapted from
\href{https://www.cs.cmu.edu/\~rwh/pfpl.html}{Practical Foundations for
Programming Languages}. This is only meant as a clarifying
touchstone.
\end{itemize}

\subsubsection{Eliminating products and introducing sums}
\label{sec:org5178260}
The elimination of products via projection and the introduction of sums
via injection are the simplest of the two pairs of rules.

\begin{enumerate}
\item Projection
\label{sec:org6c8b9ea}
Here is a concrete example of projecting a value out of a record using
our current syntax:

\begin{verbatim}
val r : {a:int} = {a:1}
val ex : int = r.a
// Or, using our exotic field operator, which is currently the normal form
val ex_ : int = r.field("a")
\end{verbatim}

A textbook rule for eliminating an expression with a finite product
types can be given as

$$
\frac
{ \Gamma \vdash e \colon \{ i_1 \hookrightarrow \tau_1, \ \ldots, \ i_n \hookrightarrow \tau_n \} \quad (1 \le k \le n)}
{ \Gamma \vdash e.i_k \colon \tau_k }
$$

Where \(i\) is drawn from a finite set of indexes used to label the
components of the product (e.g., fields of a record or positions in a
tuple) and \(i_j \hookrightarrow \tau_j\) maps the index \(i_j\) to the
corresponding type \(\tau_j\).

This rule tells us that, when an expression \(e\) with a product type is
derivable from a context, we can eliminate it by projecting out of \(e\)
with an index \(i_k\) (included in the type), giving an expression of
the type \(t_k\) corresponding to that index. If we're given a bunch of
stuff packaged together we can take out just the one part we want.

In our current system, typechecking the projection of a value out of a record
\href{https://github.com/informalsystems/quint/blob/545b14fb8c19ac71d8f08fb8500ce9cc3cabf678/quint/src/types/specialConstraints.ts\#L91-L120}{implements} the following rule

$$
\frac
{ \Gamma \vdash e \colon (r, c) \quad \Gamma \vdash `l' \colon str \quad fresh(t) }
{ \Gamma \vdash \ field(e, `l') \ \colon (t, c \land r \sim \{ \ l \colon t | tail\_t \ \}) }
$$

where

\begin{itemize}
\item we use the judgement syntax established in
\href{https://github.com/informalsystems/quint/tree/main/doc/adr005-type-system.md}{ADR5},
in which \(\Gamma \vdash e : (t, c)\) means that, in the typing
context \(\Gamma\), expression \(e\) can be derived with type \(t\)
under constraints \(c\),
\item \(fresh(t)\) is a side condition requiring the type variable \(t\) to
be fresh in \(\Gamma\),
\item \(`l'\) is a string literal with the internal representation \(l\),
\item \(c\) are the constraints derived for the type \(r\),
\item \(tail\_t\) is a free row-variable constructed by prefixing the fresh
variable \(t\) with ``tail'',
\item \(\{ \ l \colon t | tail\_t \ \}\) is the open row-based record type
with field, \(l\) assigned type \(t\) and free row- left as a free
variable,
\item and \(r \sim \{ \ l \colon t | tail\_t \ \}\) is a unification
constraint.
\end{itemize}

Comparing the textbook rule with the rule in our system helps make the
particular qualities and idiosyncrasies of our system very clear.

The most critical difference w/r/t to the complexity of the typing rules
derives form the fact that our system subordinates construction and
elimination of records to the language level operator application rather
than implementing it via a special constructs that work with product
indexes (labels) directly. This is what necessitates the consideration
of the string literal \(`l'\) in our premise. In our rule for type
checking record projections we ``lift'' quint expressions (string literals
for records and ints for products) into product indexes.

The most salient difference is the use of unification constraints. This
saves us having to ``inspect'' the record type to ensure the label is
present and obtain its type. These are both accomplished instead via the
unification of \(r\) with the minimal open record including the fresh
type \(t\), which will end up holding the inferred type for the
projected value iff the unification goes through. This feature of our
type system is of special note for our aim of introducing sum-types:
almost all the logic for ensuring the correctness of our typing
judgements is delegated to the unification rules for the row-types that
carry our fields for product type and sum types alike.

\item Injection
\label{sec:orgc77f012}
Here is a concrete example of injecting a value into a sum type:

\begin{verbatim}
val n : int = 1
val ex : A(int) = A(1)
\end{verbatim}

A textbook rule for eliminating an expression belonging to a finite
product type can be given as

$$
\frac
{ \Gamma \vdash e \colon \tau_k  \quad (1 \le k \le n)}
{ \Gamma \vdash i_k \cdot e \colon \langle i_1 \hookrightarrow \tau_1, \ \ldots, \ i_n \hookrightarrow \tau_n \rangle }
$$

Where \(i\) is drawn from a finite set of indexes used to label the
possible alternatives of the co-product and
\(i_j \hookrightarrow \tau_j\) maps the index \(i_j\) to the
corresponding type \(\tau_j\). We use \(\langle \ldots \rangle\) to
indicate the labeling is now disjunctive and \(i_k \cdot e\) as the
injection of \(e\) into the sum type using label \(i_k\). Note the
symmetry with complementary rule for projection out of a record: the
only difference is that the (now disjunctive) row (resp. (now injected)
expression) is swapped from premise to conclusion (resp. from conclusion
to premise).

This rule tells us that, when an expression \(e\) with a type \(t_k\) is
derivable from a context, we can include it as an alternative in our sum
type by injecting it with the label \(i_k\), giving an element of our
sum type. If we're given a thing that has a type allowed by our
alternatives, it can included among our alternatives.

If we were following the row-based approach outlined in
\href{https://www.microsoft.com/en-us/research/publication/extensible-records-with-scoped-labels/}{Leijen05},
then the proposed rule in our system, formed by seeking the same
symmetry w/r/t projection out from a product, would be:

$$
\frac
{ \Gamma \vdash e \colon (t, c) \quad \Gamma \vdash `l' \colon str \quad fresh(s) }
{ \Gamma \vdash \ variant(`l', e) \ \colon (s, c \land s \sim \{ \ l \colon t | tail\_s \ \}) }
$$

Comparing this with our current rule for projecting out of records, we
see the same symmetry: the (now disjunctive) row type is synthesized
instead of being taken from the context.

However, if we don't want to expose the row-polymorphism to users, we need a
more constrained rule that will ensure the free row variable is not surfaced. We
can address this by replacing the side condition requiring \(s\) to be free with a side
condition requiring that there it be defined, and in our constraint check that
we can unify that defined type with a row that contains the given label with the
expected type and is otherwise open.

$$
\frac
{
\Gamma \vdash e \colon (t, c) \quad
\Gamma \vdash `l' \colon str \quad
definedType(s) \quad
free(v)
}
{
\Gamma \vdash \ variant(`l', e) \ :
(s, c \land s \sim \langle \ l \colon t | v \ \rangle)
}
$$


Igor has voiced a strong preference that we do not allow anonymous or
row-polymorphic sum types, which is why the last rule is proposed. It does
complicate our typing rules, as it requires we draw from the typing context.
\end{enumerate}


\subsubsection{Introducing products and eliminating sums}
\label{sec:org6a0635d}
Forming expressions of product types by backing them into records and
eliminating expressions of sum types by case analysis exhibit the same
duality, tho they are a bit more complex.

\begin{enumerate}
\item Packing expressions into records
\label{sec:org254ef28}
Here is a concrete example of forming a record using our current syntax:

\begin{verbatim}
val n : int = 1
val s : str = "one"
val ex : {a : int, b : str} = {a : n, b : s}
// Or, using our exotic Rec operator, which is currently the normal form
val ex_ : {a : int, b : str} = Rec("a", n, "b", s)
\end{verbatim}

A textbook introduction rule for finite products is given as

$$
\frac
{ \Gamma \vdash e_1 \colon \tau_1 \quad \ldots \quad \Gamma \vdash e_n \colon \tau_n }
{ \Gamma \vdash \{ i_1 \hookrightarrow e_1, \ldots, i_n \hookrightarrow e_n \} \colon \{ i_1 \hookrightarrow \tau_1, \ldots, i_n \hookrightarrow \tau_n \} }
$$

This tells us that for any expressions
\(e_1 : \tau_1, \ldots, e_n : \tau_n\) derivable from our context we can
form a product that indexes those \(n\) expressions by
\(i_1, \ldots, i_n\) distinct labels, and packages all data together in
an expression of type
\(\{ i_1 \hookrightarrow \tau_1, \ldots, i_n \hookrightarrow \tau_n \}\).
If we're given all the things of the needed types, we can conjoint them
all together into one compound package.

The following rule describes our current implementation:

$$
\frac
{ \Gamma \vdash (`i_1`, e_1 \colon (t_1, c_1)) \quad \ldots \quad \Gamma \vdash (`i_1`, e_n \colon (t_n, c_n)) \quad fresh(s) }
{ \Gamma \vdash Rec(`i_1`, e_1, \ldots, `i_n`, e_n) \ \colon \ (s, c_1 \land \ldots \land c_n \land s \sim \{ i_1 \colon t_1, \ldots, i_n \colon t_n \}) }
$$

The requirement that our labels show up in the premise as quint strings
paired with each element of the appropriate type is, again, an artifact
of the exotic operator discussed later, as is the \texttt{Rec} operator in the
conclusion that consumes these strings. Ignoring those details, this
rule is quite similar to the textbook rule, except we use unification of
the fresh variable \texttt{s} to propagate the type of the constructed record,
and we have to do some bookkeeping with the constraints from each of the
elements that will be packaged into the record.

\item Performing case analysis
\label{sec:org0e6979d}
Here is a concrete example of case analysis to eliminate an expression
belonging to a sum type using the proposed syntax variants:

\begin{verbatim}
val e : T = A(1)
def describeInt(n: int): str = if (n < 0) then "negative" else "positive"
val ex : str = match e {
  | A(x) => describeInt(x)
  | B(x) => x
}
\end{verbatim}

A textbook rule for eliminating an expression that is a variant of a
finite sum type can be given as

$$
\frac{
\Gamma \vdash e \colon 
\langle i_1 \hookrightarrow \tau_1, \ldots, i_n \hookrightarrow \tau_n \rangle 
\quad 
\Gamma, x_1 : \tau_1 \vdash e_1 \colon \tau
\quad 
\ldots
\quad 
\Gamma, x_n : \tau_n \vdash e_n \colon \tau
}
{ \Gamma \vdash \ match \ e \ 
\{ i_1 \cdot x_1 \hookrightarrow e_1 | \ldots | i_n \cdot x_n \hookrightarrow e_n \} : \tau }
$$

Note the complementary symmetry compared with the textbook rule for
product construction: product construction requires \texttt{n} expressions to
conclude with a single record-type expression combining them all; while
sum type elimination requires a single sum-typed expression and \texttt{n} ways
to convert each of the \texttt{n} alternatives of the sum type to conclude with
a single expression of a type that does not need to appear in the sum
type at all.

The proposed rule for quint's type system is given without an attempt to
reproduce our practice of using quint strings. This can be added in
later if needed:

$$
\frac{
\Gamma \vdash e : (s, c) \quad
\Gamma, x_1 \vdash e_1 : (t, c_1) \quad \ldots \quad \Gamma, x_n \vdash e_n : (t, c_n) \quad
\Gamma, \langle v \rangle \vdash e_{n+1} : (t, c_{n+1}) \quad
fresh(v)
}{
\Gamma \vdash \ match \ e \ \{ i_1 : x_1 \Rightarrow e_1, \ldots, i_n : x_n \Rightarrow e_n \} : (t,
c \land c_1 \land \ldots \land c_n \land c_{n+1} \land
s \sim \langle i_1 : t_1, \ldots, i_n : t_n | v \rangle)
}
$$

Compared with quint's rule for product construction we see the same
complementary symmetry. However, we also see a striking difference:
there is no row variable occurring in the product construction, but the
row variable plays an essential function in sum type elimination of
row-based variants. Row types in records are useful for extension
operations (i.e., which we don't support in quint currently) and for
operators that work over some known fields but leave the rest of the
record contents variable. But the core idea formalized in a product type
is that the constructor \emph{must} package all the specified things together
so that the recipient \emph{can} chose any thing; thus, when a record is
constructed we must supply all the things and there is no room for
variability in the row. For sum types, in contrast the constructor \emph{can}
supply any one thing (of a valid alternate type), and requires the
recipient \emph{must} be prepared to handle every possible alternative.

In the presence of row-polymorphis, however, the responsibility of the
recipient is relaxed: the recipient can just handle a subset of the
possible alternatives, and if the expression falls under a label they
are not prepared to handle, they can pass the remaining responsibility
on to another handler.

Here is a concrete example using the proposed syntax, note how we narrow
the type of \texttt{T}:

\begin{verbatim}
type T = A | B;;
def f(e) = match e {
  | A => 1
  | B => 2
  | _ => 0
}

// f can be applied to a value of type T
let a : T = A
let ex : int = f(A)

// but since it has a fallback for an open row, it can also handle any other variant
let foo = Foo
let ex_ : int = f(foo)
\end{verbatim}

Here's the equivalent evaluated in OCaml as proof of concept:

\begin{verbatim}
utop # 
type t = [`A | `B]
let f = function
  | `A -> 1
  | `B -> 2
  | _  -> 0
let ex = f `A, f `Foo
;;
type t = [ `A | `B ]
val f : [> `A | `B ] -> int = <fun>
val ex : int * int = (1, 0)
\end{verbatim}

All the features just discussed that come from row-polymorphism will not be
available since we are choosing to suppress the row-typing.
\end{enumerate}

\subsection{Concrete Syntax}
\label{sec:orgb377490}

\subsubsection{Why not support ``polymorphic variants''}
\label{sec:org4566b03}
Our sum type system is based on row-polymorphism. The only widely used language
I've found that uses row-polymorphism for extensible sum types is OCaml (and the
alternative surface syntax, ReScript/Reason). For examples of their syntax for
this interesting construct, see

\begin{description}
\item[{ReScript}] \url{https://rescript-lang.org/docs/manual/latest/polymorphic-variant}
\item[{OCaml}] \url{https://v2.ocaml.org/manual/polyvariant.html}
\end{description}

These very flexible types are powerful, but they introduce challenges to the
syntax (and semantics) of programs. For example, supporting anonymous variant
types requires a way of constructing variants without pre-defined constructors.
Potential approaches to address this include:

\begin{itemize}
\item A special syntax that (ideally) mirrors the syntax of the type.
\item A special lexical marker on the labels (what ReScrips and OCaml do),
e.g., \texttt{`A 1} or \texttt{\#a 1} instead of \texttt{A(1)}.
\end{itemize}

However, we have instead opted to hide the row-polymorphism, and not expose
this.

\subsubsection{Declaration}
\label{sec:org9d13082}

\begin{enumerate}
\item Why use the \texttt{|} syntax to separate alternatives?
\label{sec:org2585d56}

\begin{itemize}
\item In programming \texttt{|} is \href{https://en.wikipedia.org/wiki/Vertical\_bar\#Disjunction}{widely used for disjunction}:
\begin{itemize}
\item regex
\item boolean ``or''
\item bitwise ``or''
\item alternatives in BNF
\item parallel execution on the pi-calculus
\end{itemize}
\item Many modern languages with support for sum types (or the more general union
types) use \texttt{|}, including
\begin{itemize}
\item \href{https://docs.python.org/3/library/typing.html\#typing.Union}{Python}
\item \href{https://www.typescriptlang.org/docs/handbook/2/everyday-types.html\#union-types}{TypeScript}
\item (of course) the MLs, Haskell, F\#, Elm, OCaml, etc.
\end{itemize}
\end{itemize}

\item Why not use \texttt{,} to separate alternatives?
\label{sec:org36d828e}

We have discussed modeling our concrete syntax for sum type declarations on
Rust. But without changes to other parts of our language, this would leave the
concrete syntax for type declarations too similar to record type declarations.

This similarity is aggravated by the fact that we currently don't enforce any
case-based syntactic rules to differentiate identifiers used for  operator
names, variables, or module names, and we are currently planning to extend this
same flexibility to variant labels, just as we do for record field labels.
Thus, we could end up with a pair of declarations like:

\begin{verbatim}
type T = {
    A : int,
    b : str,
}

type S = {
    A(int),
    b(str),
}
\end{verbatim}

We are not confident that the difference between \texttt{\_:\_} and \texttt{\_(\_)} will be enough
to keep readers from confusing the two.

But the chance of mistaking a record and sum type declaration is actually
compounding a worse possible confusion: the part of a sum-type record enclosed
in brackets is syntactically indistinguishable from a block of operators
applications.

Given we tend to read data structures from the outside in, we feel confident
that we were going to avoid confusion by requiring declarations to use \texttt{|} to
demarcate alternatives:

\begin{verbatim}
type T = {
    a : int,
    B : str,
}

type S =
    | A(int)
    | b(str)
\end{verbatim}

The latter seems much clearer to our team, and if we reflect this syntax also in
\texttt{match}, it will give another foothold to help readers gather meaning when
skimming the code.


\item Could we just copy Rust exactly?
\label{sec:org9e9a2cd}

In Rust, sum types are declared like this:

\begin{verbatim}
enum T {
  A(i64),
  B(i64),
  C
}
\end{verbatim}

If we just adopted this syntax directly without also changing our syntax for
records, we would introduce

\begin{itemize}
\item Breaks with our current convention around type declarations and
use of keywords.
\item May mislead users to try injecting values into the type via Rust's
\texttt{T::A(x)} syntax, which clashes with our current module syntax.
\item This would move us closer to Rust but further from languages like \href{https://www.typescriptlang.org/docs/handbook/2/everyday-types.html\#type-aliases}{TypeScript}
and Python.
\end{itemize}

Rust't syntax makes pretty good sense \uline{within the context of the rest of Rust's
syntax}. Here is an overview:

\textbf{declaration}

\begin{verbatim}
struct Pair {
    fst: u64,
    snd: String
}

enum Either {
    Left(u64),
    Right(String)
}
\end{verbatim}

\textbf{construction}

\begin{verbatim}
let s = Either::Left(4)
let p = Pair{
    fst: 4,
    snd: "Two"
}
\end{verbatim}

\textbf{elimination}

\begin{verbatim}
let two = p.snd
let four = s match {
    Either::Left(_) => 4,
    Either::Right(_) => 4
}
\end{verbatim}

Note how the various syntactic elements work together to give consistent yet
clearly differentiated forms of expression for dual constructs:

\begin{itemize}
\item The prefix keyword for declarations consistently (\texttt{enum} vs. \texttt{struct}).
\item A unique constructor name is used consistently for forming a record or
variant.
\item Lexical rules add clear syntactic markers:
\begin{itemize}
\item Data constructors must begin with a capital letter
\item Field names of a struct (and method names of a trait, and module names) must
\end{itemize}
begin with a lowercase letter.
\begin{itemize}
\item This ensures the syntax for module access and sum type construction are
unambiguous: \texttt{mymod::foo(x)} vs \texttt{MySumType::Foo(x)}.
\item The caps/lower difference also helps reflect the duality between record
fields and sum type alternatives.
\end{itemize}
\end{itemize}

Compare with the syntax proposed for quint this RFC:

\textbf{declaration}

\begin{verbatim}
type Pair = {
  fst: int,
  snd: str,
}

type Either =
  | left(int),
  | right(str)
\end{verbatim}

\textbf{construction}

\begin{verbatim}
let s = left(4)
let p = {
    fst: 4,
    snd: "Two"
}
\end{verbatim}

\textbf{elimination}

\begin{verbatim}
let two = p.snd
let four = s match {
    | left(_) => 4
    | right(_) => 4
}
\end{verbatim}

Following the current quint syntax, we don't differentiate declaration with
keywords or data construction with prefix use of type names. But we make up for
this lost signal with the evident difference between \texttt{|} and \texttt{,}. This also
compensates for the inability to differentiate based on capitalization.

Finally, by adopting a syntax that if very similar to C++-like languages (like
Rust), we risk presenting false friends. There are numerous subtle differences
between quint and Rust, and if we lull users into thinking the syntax is
roughly the same, they are likely to be disappointed when they discover that, e.g.,

\begin{itemize}
\item Unlike Rust, conditions in \texttt{if} must be wrapped in \texttt{(...)}
\item Unlike Rust, conditions must have an \texttt{else} branch
\item Unlike Rust, \texttt{let} is not used for binding
\item Unlike Rust, type parameters are surrounded in \texttt{[...]} rather than \texttt{<...>}
\item Unlike Rust, operator are declared with \texttt{def}
\item Unlike Rust, type variables must begin with a lower-case letter
\end{itemize}

In short, given how many ways we differ from Rust syntax already, adopting
Rust's syntax for sum types would be confusing in the context of our current
suntax and possibly lead to incorrect expectations.

\item What if we want to be more Rust-like
\label{sec:org714fa9b}

If we want to use a syntax for sum types that is closer to, or exactly the same
as, Rust's, then we should make at least the following changes to the rest of
our syntax to preserve harmony:

\begin{itemize}
\item Require a prefix name relating to a type when constructing data, e.g., \texttt{Foo
  \{a: 1, b: 2\}} for constructing a record of type \texttt{Foo}. (Note this would mean
dropping support for anonymous records types.)
\item Introduce a lexical distinction between capitalized identifiers, used for
data constructors (and type constants), and uncapitalized identifiers, used
for records field labels and operators.
\item Use keywords consistently for type declarations.
\end{itemize}

Taking these changes into account, we could render the previous quint examples
thus:

\textbf{declaration}

\begin{verbatim}
type Pair = struct {
  fst: u64,
  snd: String
}

type Either = enum {
  Left(u64),
  Right(String)
}
\end{verbatim}

\textbf{construction}

\begin{verbatim}
let s = Left(4)
let p = Pair {
    fst: 4,
    snd: "Two"
}
\end{verbatim}

\textbf{elimination}

\begin{verbatim}
let two = p.snd
let four = s match {
    Left(_) => 4,
    Right(_) => 4,
}
\end{verbatim}
\end{enumerate}


\subsubsection{Case analysis}
\label{sec:org1404ef7}
Ergonomic support for sum types requires eliminators, ideally in the
form if case analysis by pattern matching.

The proposed syntax is close to \href{https://doc.rust-lang.org/book/ch18-03-pattern-syntax.html\#matching-literals}{Rust's pattern syntax}, modulo swapping \texttt{|}
for \texttt{,} to be consistent with the type declaration syntax.

Here's some example Rust for comparison

\begin{verbatim}
    match x {
        A    => println!("a"),
        B    => println!("b"),
        C(v) => println!("cv"),
        _    => println!("anything"),
    }
\end{verbatim}

The \texttt{match} is a close analogue to our existing \texttt{if} expressions, and
the reuse of the \texttt{=>} hints at the connection between case elimination
and anonymous operators. The comma separated alternatives enclosed in
\texttt{\{...\}} follow the variadic boolean action operators, which seems
fitting, since sum types are disjunction over data.

One question if we adopt some form of pattern-based case analysis is how far we
generalize the construct. Do we support pattern matching on scalars like ints
and symbols? Do we support pattern matching to deconstruct compound data such as
records and lists? What about sets? Do we allow pattern expressions to serve as
anonymous operator (like Scala)?

My guess is that in most cases the gains in expressivity of specs would justify
the investment, but it is probably best to start with limiting support to
defined sum types and seeing where we are after that.

Until we have pattern matching introduced, we should flag a parsing error if the
deconstructor argument is not a free variable, and inform the user that full pattern
matching isn't yet supported.

\subsubsection{Sketch of an alternative syntax}
\label{sec:org726607c}
The syntax being proposed is chosen because it is familiar to Rust programmers,
and is deemed sufficient so long as we don't need to expose the underlying row
polymorphism. However, it has the down-sides of being very similar to the syntax
for records, which might lead to confusion. I've also considered a more distinctive alternative which
is also more consistently complementary to our records syntax. This group of alternatives follows \href{https://www.microsoft.com/en-us/research/publication/extensible-records-with-scoped-labels/}{Leijen05}:

\begin{enumerate}
\item Declaration
\label{sec:org78427f0}

Reflecting the fact that both records and sum types are based on rows, we
use the same pairing (\texttt{:}) and enumerating (\texttt{,}) syntax, but signal to move from
a conjunctive to a disjunctive meaning of the row by changing the brackets:

\begin{verbatim}
type T =
  < A : int
  , B : str
  >
\end{verbatim}

\item Injection
\label{sec:orgd20c363}

Injection uses a syntax that is dual with projection on records: \texttt{.} projects
values out of products and injects them into co-products:

\begin{verbatim}
val a : T = <A.1>
\end{verbatim}

Since this option gives a syntactically unambiguous representation of
variant formation, there is no need to generate special injector operator, and
\texttt{<\_.\_>} can be the normal form for injection.

Annotation of anonymous sum types is clear and unambiguous:

\begin{verbatim}
def f(n: int): <C:int, D:str | s> =
  if (n >= 0) <C.n> else <D."negative">
\end{verbatim}

Compare with the corresponding annotation for a record type:

\begin{verbatim}
def f(n: int): {C:int, D:str | s} =
  if (n >= 0) {C:n, D:"positive"} else {C:n, B:"negative"}
\end{verbatim}

\item Elimination
\label{sec:org84c35ee}

Finally, elimination uses a syntax that is dual to record construction,
signaling the similarity thru use of the surrounding curly braces, and
difference via the presence of the fat arrows (this syntax is similar to the
one proposed):

\begin{verbatim}
match e {
  A : a => ...,
  B : b => ...
}
\end{verbatim}
\end{enumerate}

\subsection{High-level implementation plan}
\label{sec:orgd09f6a5}

\begin{description}
\item[{Add parsing and extension of the IR for the syntax}] \url{https://github.com/informalsystems/quint/pull/1092}
\item[{Add generation of constructor operators}] \url{https://github.com/informalsystems/quint/pull/1093}
\item[{Add rules for type checking}] \url{https://github.com/informalsystems/quint/issues/244}
\item[{Add support in the simulator}] \url{https://github.com/informalsystems/quint/issues/1033}
\item[{Add support for converting to Apalache}] \url{https://github.com/informalsystems/quint/issues/1034}
\end{description}

\subsection{Additional consideration}
\label{sec:org0d2407a}
\subsubsection{Pattern matching}
\label{sec:orgbddbca7}
We may want to consider support for pattern matching at some point. I suspect
the \texttt{match} construct will make users want this, but then again perhaps quint
is simple enough that we can do without this complication. We'd also have to
consider carefully how this would work with conversion to the Apalache IR.

\subsubsection{User defined parametric type constructors}
\label{sec:org1a4beae}

If we want to gain the most value from the addition of sum types we should allow
users to define parametric types such as \texttt{Option} and \texttt{Either}. See
\url{https://github.com/informalsystems/quint/issues/1073} .

\subsubsection{Drop the exotic operators}
\label{sec:orgd5aba72}
\begin{itemize}
\item Remove the special product type operators \texttt{fieldNames}, \texttt{Rec}, \texttt{with},
\texttt{label}, and \texttt{index}, or add support for first-class labels As is, I
think these are not worth the complexity and overhead.
\end{itemize}

Compare our rule with the projection operation from ``Extensible Records
with Scoped Labels'', which does not receive the label `l' as a string,
instead treating it as a special piece of syntax:

\begin{verbatim}
(_.l) :: ∀r α. (l|r) ⇒ {l :: α | r } → α`
\end{verbatim}

Another point of comparison is Haskell's
\href{https://www.haskell.org/onlinereport/haskell2010/haskellch3.html\#x8-490003.15}{``Datatypes
with Field Labels''}, which generates a projection function for each
label, so that defining the datatype

\begin{verbatim}
data S = S1 { a :: Int, b :: String }
\end{verbatim}

will produce functions

\begin{verbatim}
a :: S -> Int
b :: S -> String
\end{verbatim}

\begin{enumerate}
\item Benefits
\label{sec:org36441ad}

\begin{enumerate}
\item Simplified typing rules
\label{sec:org0e97219}
Abandoning this subordination to normal operator application would leave
us with a rule like the following for record projection:

$$
\frac
{ \Gamma \vdash e \colon (r, c) \quad fresh(t) }
{ \Gamma \vdash \ e.l \ \colon (t, c \land r \sim \{ \ l \colon t | tail\_t \ \}) }
$$

This would allow us to remove the checks for string literals, instead leaving that
to the outermost, syntactic level of our static analysis. A similar
simplification would follow for record construction: the rule for \texttt{Rec} would
not need to validate that it had received an even number of argument of
alternating string literals and values, since this would be statically
guaranteed by the parsing rules for the \(\{ l_1 : v_1, \ldots, l_n : v_n \}\)
syntax. This would be a case of opting for the \href{https://lexi-lambda.github.io/blog/2019/11/05/parse-don-t-validate/}{``Parse, don't validate''} strategy.

\item Safer language
\label{sec:org53a2258}

The added surface area introduced by these operators have contributed to several
bugs, including at least the ones discussed here:

\begin{itemize}
\item \url{https://github.com/informalsystems/quint/issues/169}
\item \url{https://github.com/informalsystems/quint/issues/816}
\item \url{https://github.com/informalsystems/quint/issues/1081}
\end{itemize}

\item More maintainable code base
\label{sec:org03c9b45}

We are losing structure in our internal representation of expressions, which
means we get less value out of typescript type system, have to do more work
when converting to the Apalache IR, and have to deal with more fussy details. By
converting the rich structures of records into an internal representation
\texttt{Rec(expr1, ..., expr1)}

\begin{itemize}
\item we lose the ability to ensure we are forming records correctly statically
\item have to do manual arithmetic to ensure fields and values are paired up
correctly
\item essentially reduce ourselves to a ``stringly typed'' representation, relying
entirely on detecting the \texttt{Rec} value in the \texttt{kind} field.
\end{itemize}
\end{enumerate}
\end{enumerate}
\end{document}